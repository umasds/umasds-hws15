\documentclass{beamer}
\usepackage[T1]{fontenc}
\usepackage[utf8]{inputenc}
\usepackage[ngerman]{babel} 
\usepackage{fancyvrb}
\usepackage{amsmath}
\usepackage{amssymb}
\usepackage{libertine}
\usepackage{courier}
\usepackage{hyperref}
%\usepackage[style=authoryear,backend=biber]{biblatex}
\usepackage[babel]{csquotes}
\usepackage{booktabs}
\usepackage{multirow}
%\addbibresource{literatur.bib}
%\renewcommand*{\bibfont}{\footnotesize}

\makeatletter
\def\url@foostyle{%
  \@ifundefined{selectfont}{\def\UrlFont{\sf}}{\def\UrlFont{\footnotesize\ttfamily}}}
\makeatother
\urlstyle{sf}

\usepackage{tikz}
\usetikzlibrary{shapes,arrows,decorations.pathmorphing,backgrounds,positioning,fit,petri}
\tikzstyle{latent}=[circle,draw=black,semithick,inner sep=0pt,minimum size=10mm]
\tikzstyle{manifest}=[rectangle,draw=black,semithick,inner sep=0pt,minimum size=10mm]
\tikzstyle{const}=[regular polygon,regular polygon sides=3,draw=black!30,semithick,inner sep=0pt,minimum size=10mm]
\tikzstyle{resid}=[inner sep=0pt,minimum size=4mm]
\tikzstyle{corr}=[<->,shorten >=1pt,shorten <=1pt,>=stealth',semithick]
\tikzstyle{coef}=[->,shorten >=1pt,shorten <=1pt,>=stealth',semithick]
%\tikzset{venn circle/.style={draw,circle,minimum width=2cm,fill=#1,opacity=0.6}}
\tikzset{venn circle/.style={draw,circle,minimum width=2cm}}

%-------------------------------------------------------------------
\beamertemplatenavigationsymbolsempty
\makeatletter
\setbeamertemplate{footline}{
  \leavevmode%
  \hbox{%
  \begin{beamercolorbox}[wd=.333333\paperwidth,ht=.5cm,dp=1ex,left]
    {author in head/foot}%
    \hspace{.05cm}
    \includegraphics[height=0.15cm]{../../img/80x15.png}
    \hspace{.05cm}
    \usebeamerfont{author in head/foot}\insertshortauthor
  \end{beamercolorbox}%
  \begin{beamercolorbox}[wd=.666666\paperwidth,ht=.5cm,dp=1ex,right]
    {date in head/foot}%
    \vspace{-.1cm}
    \includegraphics[height=0.3cm]{../../img/150px-Uni_Mannheim_Siegel.png}
    \hspace{.05cm}
    \includegraphics[height=0.3cm]{../../img/sswml_logo2.png}
  \end{beamercolorbox}}%
  \vskip0pt%
}
\makeatother

%-------------------------------------------------------------------
\RecustomVerbatimEnvironment
  {Verbatim}{Verbatim}
  {frame=single,gobble=4,formatcom=\color{blue!50!black},}
%fontsize=\footnotesize
%-------------------------------------------------------------------
\title{Forschungsseminar -- Social Data Science}
\subtitle{Ablauf}
\author[Beste \& Bethmann]{Jonas Beste und Arne Bethmann}
\date{\includegraphics[height=1.5cm]{../../img/150px-Uni_Mannheim_Siegel.png} \\
  \href{http://sswml.uni-mannheim.de/}{Professorship for Statistics \& Methodology} \\
  HWS 2015}
  
%-------------------------------------------------------------------
\begin{document}

%-------------------------------------------------------------------
\begin{frame}
  \titlepage
\end{frame}


%-------------------------------------------------------------------
\begin{frame}
  \frametitle{Termine}
  \begin{itemize}
  \item Fr, 9.10.
    \begin{description}
    \item[09:00--10:30h] Einführung in die virtuelle Maschine und Git
    \item[11:00--12:30h] Einführung in R und RStudio
    \item[Mittagspause]
    \item[13:30--15:00h] Grundlegende Verfahren des Machine Learning
    \item[15:30--17:00h] Anwendung von Machine Learning in R
    \end{description}
  \item Sa, 10.10.
    \begin{description}
    \item[09:00--10:30h] Einführung in \LaTeX
    \item[11:00--12:30h] Grafiken in R
    \item[Mittagspause]
    \item[13:30--15:00h] Diskussion von möglichen Datensätzen
    \item[15:30--17:00h] Team- und Themenfindung
    \end{description}
  \item 4./5.12. -- Präsentation Projektzwischenstand
  \end{itemize}
\end{frame}


%-------------------------------------------------------------------
\begin{frame}
\frametitle{Projektvorgaben für Prüfungsleistung}
\begin{enumerate}
\item Thema aus einem der Bereiche
  \begin{itemize}
  \item Familienforschung
  \item Bildungsforschung
  \item Arbeitsmarktforschung
  \end{itemize}
\item Verwendung von Machine Learning-Techniken
\item Reproduzierbare Forschung
  \begin{itemize}
  \item Text in {\LaTeX} (5000 bis 8000 Worte -- harte Grenzen!)
  \item Analysen und Grafiken in R
  \item Versionierung und Kollaboration mit Git
  \item Präzise Zitierung des Datensatzes
  \item Lauffähige Skripte um aus Rohdaten Seminarabeit zu erzeugen
  \end{itemize}
\item Theoretische Verankerung im einschlägigen Forschungsstand
\end{enumerate}
\end{frame}


%-------------------------------------------------------------------
\end{document}
