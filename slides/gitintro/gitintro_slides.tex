\documentclass{beamer}
\usepackage[T1]{fontenc}
\usepackage[utf8]{inputenc}
\usepackage[ngerman]{babel} 
\usepackage{fancyvrb}
\usepackage{amsmath}
\usepackage{amssymb}
\usepackage{libertine}
\usepackage{courier}
\usepackage{hyperref}
%\usepackage[style=authoryear,backend=biber]{biblatex}
\usepackage[babel]{csquotes}
\usepackage{booktabs}
\usepackage{multirow}
%\addbibresource{literatur.bib}
%\renewcommand*{\bibfont}{\footnotesize}

\makeatletter
\def\url@foostyle{%
  \@ifundefined{selectfont}{\def\UrlFont{\sf}}{\def\UrlFont{\footnotesize\ttfamily}}}
\makeatother
\urlstyle{sf}

\usepackage{tikz}
\usetikzlibrary{shapes,arrows,decorations.pathmorphing,backgrounds,positioning,fit,petri}
\tikzstyle{latent}=[circle,draw=black,semithick,inner sep=0pt,minimum size=10mm]
\tikzstyle{manifest}=[rectangle,draw=black,semithick,inner sep=0pt,minimum size=10mm]
\tikzstyle{const}=[regular polygon,regular polygon sides=3,draw=black!30,semithick,inner sep=0pt,minimum size=10mm]
\tikzstyle{resid}=[inner sep=0pt,minimum size=4mm]
\tikzstyle{corr}=[<->,shorten >=1pt,shorten <=1pt,>=stealth',semithick]
\tikzstyle{coef}=[->,shorten >=1pt,shorten <=1pt,>=stealth',semithick]
%\tikzset{venn circle/.style={draw,circle,minimum width=2cm,fill=#1,opacity=0.6}}
\tikzset{venn circle/.style={draw,circle,minimum width=2cm}}

%-------------------------------------------------------------------
\beamertemplatenavigationsymbolsempty
\makeatletter
\setbeamertemplate{footline}{
  \leavevmode%
  \hbox{%
  \begin{beamercolorbox}[wd=.333333\paperwidth,ht=.5cm,dp=1ex,left]
    {author in head/foot}%
    \hspace{.05cm}
    \includegraphics[height=0.15cm]{80x15.png}
    \hspace{.05cm}
    \usebeamerfont{author in head/foot}\insertshortauthor
  \end{beamercolorbox}%
  \begin{beamercolorbox}[wd=.666666\paperwidth,ht=.5cm,dp=1ex,right]
    {date in head/foot}%
    \vspace{-.1cm}
    \includegraphics[height=0.3cm]{150px-Uni_Mannheim_Siegel.png}
    \hspace{.05cm}
    \includegraphics[height=0.3cm]{sswml_logo2.png}
  \end{beamercolorbox}}%
  \vskip0pt%
}
\makeatother

%-------------------------------------------------------------------
\RecustomVerbatimEnvironment
  {Verbatim}{Verbatim}
  {frame=single,gobble=4,formatcom=\color{blue!50!black},}
%fontsize=\footnotesize
%-------------------------------------------------------------------
\title{Forschungsseminar -- Social Data Science}
\subtitle{Git Intro}
\author[Beste \& Bethmann]{Jonas Beste und Arne Bethmann}
\date{\includegraphics[height=1.5cm]{150px-Uni_Mannheim_Siegel.png} \\
  \href{http://sswml.uni-mannheim.de/}{Professorship for Statistics \& Methodology} \\
  HWS 2015}
  
%-------------------------------------------------------------------
\begin{document}

%-------------------------------------------------------------------
\begin{frame}
  \titlepage
\end{frame}


%-------------------------------------------------------------------
\begin{frame}
  \frametitle{Was ist ein Repository?}
  \begin{itemize}
    \item Eine Art Datenbank
    \item Protokolliert Veränderungen an Dateien
    \item Speichert neben Inhalten auch Zusatzinformationen: z.~B Wer und Wann
    \item Bezieht sich auf ein Verzeichnis in dem man Dateien für ein Projekt abgelegt hat und bearbeitet
    \item Gibt es \emph{lokal} auf dem eigenen Rechner und \emph{remote} z.~B. bei GitHub
  \end{itemize}
\end{frame}


%-------------------------------------------------------------------
\begin{frame}
  \frametitle{Der Workflow}
  \begin{block}{Repository erstmalig lokal anlegen}
  \begin{enumerate}
    \item Klonen des Remote-Repositories um eine lokale Kopie anzulegen (\texttt{git clone})
    \begin{itemize}
    \item Dadurch wir sowohl eine Kopie der Datenbank (versteckt, Unterverzeichnis \texttt{.git}) angelegt,
    \item als auch die Dateien und Verzeichnisse des Repositories lokal angelegt (Arbeitskopie aka working copy)
    \end{itemize}
  \end{enumerate}
  \begin{itemize}
  \item Alternativ kann man auch ein ganz neues lokales Repository anlegen ohne auf ein Remote-Repository zurückzugreifen (\texttt{git init})
  \end{itemize}
  \end{block}
\end{frame}



%-------------------------------------------------------------------
\begin{frame}
  \frametitle{Der Workflow}

  \begin{block}{Eigene Änderungen einpflegen}
  \begin{enumerate}
    \item Eine oder mehrere Dateien bearbeiten
    \item Einige oder alle der geänderten Dateien zum Einpflegen ins lokale Repository vormerken (aka staging, \texttt{git add})
    \item Einpflegen der Änderungen an den vorgemerkten Dateien in das lokale Repository (\texttt{git commit})
      \begin{itemize}
      \item Dazu muss ein Kommentar eingegeben werden, der möglichst knapp aber verständlich die Änderungen beschreibt
      \item Es empfiehlt sich immer miteinander zusammenhängende Änderungen zu einem commit zusammenzufassen
      \end{itemize}
    \item Die Änderungen am lokalen Repository mit dem Remote-Repository abgleichen um die Änderungen für alle verfügbar zu machen (\texttt{git push})
  \end{enumerate}
  \end{block}

\end{frame}



%-------------------------------------------------------------------
\end{document}
